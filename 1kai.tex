\documentclass[a4paper, 11pt]{ltjsarticle}
\usepackage{base, math_phys, book_theorem}

\begin{document}
\begin{center}
    {\large 情報機器およびデジタル教材の活用 課題1}
\end{center}
\begin{flushright}
    22B41185 橋本龍徳
\end{flushright}
メイヤー氏は重要なポイントを次のように結論づけていた。
\begin{enumerate}
    \item マルチメディアの利用
    \item 直観の促進
    \item 問いは短くする
    \item 生徒自身に問題を定式化させる
    \item 教師側は手助けを減らす
\end{enumerate}
これらから分かることは、数学の授業でも問題解決能力の育成が重要であるということである。それこそが深い学びである。
この深い学びを達成するために、普段の授業から対話的な学びと主体的な学びを実施する必要があると主張していると整理できるだろう。

ではなぜ数学だけがこうした問題に直面しているのか。これについてメイヤー氏は、数学の現実問題との繋がりの薄さを指摘している。実際の問題でないからこそ、ただ公式を当てはめるだけになっていたり、教授中心の授業になりがちであることが挙げられる。加えて、公式に当てはめるだけになっていることが問題点である。すなわち、汎化出来ていないか、転移できていないのである。しかしながら、数学も現実問題に向き合うためのものである。

メイヤー氏は各授業の設計の工夫についても言及していた。
各授業は、現実問題に即した設問を小ステップに分解して、問題を解いてゆくようにする。問題を与える際にはマルチメディアなどを使って現実味を帯びさせ、直観を誘導する。
これはまさにプログラム学習におけるスモールステップの原理とARCSの利用である。
これにより現実問題となり、どのように問題に向き合えば良いか分かるようになる。数学に抵抗を持つ生徒でも、ARCSによる動機づけと直観の誘導によって、対話的な学びに参加することができる。そして、どうしたら問題を解くことができるかを周囲との対話によって検討できるようになる。問題解決の流れを学ぶということは、新たな見方・考え方を獲得することと言って良いだろう。

前段落前半で述べたことは特に主体的な学び、後半は対話的な学びに対応する。主体的な学びは特に、転移に関わる部分である。公式がなぜ必要なのか、そしてどうしてその公式は公式たるのかの部分を習得することができるということである。対話的な学びでは、自分だけでは知り得ない情報や別の見方・考え方を獲得できるという大きな意味がある。

深い学びとは、自分自身でそれを活用して考えられるようになる学びである。数学を使って問題解決できるようになることである。そのためには(i)〜(v)を行うことがか求められる。

\end{document}